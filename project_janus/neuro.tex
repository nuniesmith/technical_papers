\documentclass[12pt, a4paper]{article}

% --- PACKAGES ---
\usepackage[utf8]{inputenc}
\usepackage[T1]{fontenc}
\usepackage{pmboxdraw}
\usepackage{newunicodechar}
\usepackage[english]{babel}
\usepackage{helvet}
\renewcommand{\familydefault}{\sfdefault}
\usepackage{setspace}
\usepackage[top=2.5cm, bottom=2.5cm, left=2.5cm, right=2.5cm]{geometry}
\usepackage{amsmath, amssymb, amsfonts}
\usepackage{graphicx}
\usepackage{xcolor}
\usepackage{fancyhdr}
\usepackage{titlesec}
\usepackage{enumitem}
\usepackage{listings}
\usepackage{tcolorbox}
\usepackage{tabularx}
\usepackage{array}
\usepackage{algorithm}
\usepackage{algpseudocode}
\usepackage{mathtools}

% --- Define Left-Aligned X Column for Tables ---
\newcolumntype{L}{>{\raggedright\arraybackslash}X}

% --- URL BREAKING ---
\usepackage{xurl}
\usepackage{hyperref}

% --- CONFIGURATION ---
\onehalfspacing
\setlength{\headheight}{15pt}

\definecolor{janusblue}{RGB}{0, 51, 102}
\definecolor{accentgold}{RGB}{204, 153, 51}
\definecolor{codegray}{RGB}{245, 245, 245}
\definecolor{neurocolor}{RGB}{70, 130, 180}
\definecolor{braincolor}{RGB}{100, 149, 237}

% --- UNICODE CHARACTER DECLARATIONS ---
\newunicodechar{▼}{\ensuremath{\blacktriangledown}}
\newunicodechar{→}{\ensuremath{\rightarrow}}
\newunicodechar{←}{\ensuremath{\leftarrow}}
\newunicodechar{↔}{\ensuremath{\leftrightarrow}}
\newunicodechar{⇒}{\ensuremath{\Rightarrow}}
\newunicodechar{…}{\ldots}
\newunicodechar{≥}{\ensuremath{\geq}}
\newunicodechar{≤}{\ensuremath{\leq}}
\newunicodechar{≠}{\ensuremath{\neq}}
\newunicodechar{≈}{\ensuremath{\approx}}
\newunicodechar{∈}{\ensuremath{\in}}
\newunicodechar{∉}{\ensuremath{\notin}}
\newunicodechar{∧}{\ensuremath{\wedge}}
\newunicodechar{∨}{\ensuremath{\vee}}
\newunicodechar{¬}{\ensuremath{\neg}}
\newunicodechar{×}{\ensuremath{\times}}
\newunicodechar{÷}{\ensuremath{\div}}
\newunicodechar{∞}{\ensuremath{\infty}}
\newunicodechar{∑}{\ensuremath{\sum}}
\newunicodechar{∏}{\ensuremath{\prod}}
\newunicodechar{∫}{\ensuremath{\int}}
\newunicodechar{√}{\ensuremath{\sqrt}}
\newunicodechar{∂}{\ensuremath{\partial}}
\newunicodechar{∇}{\ensuremath{\nabla}}
\newunicodechar{α}{\ensuremath{\alpha}}
\newunicodechar{β}{\ensuremath{\beta}}
\newunicodechar{γ}{\ensuremath{\gamma}}
\newunicodechar{δ}{\ensuremath{\delta}}
\newunicodechar{ε}{\ensuremath{\epsilon}}
\newunicodechar{θ}{\ensuremath{\theta}}
\newunicodechar{λ}{\ensuremath{\lambda}}
\newunicodechar{μ}{\ensuremath{\mu}}
\newunicodechar{π}{\ensuremath{\pi}}
\newunicodechar{σ}{\ensuremath{\sigma}}
\newunicodechar{τ}{\ensuremath{\tau}}
\newunicodechar{φ}{\ensuremath{\phi}}
\newunicodechar{ω}{\ensuremath{\omega}}
\newunicodechar{Δ}{\ensuremath{\Delta}}
\newunicodechar{Σ}{\ensuremath{\Sigma}}
\newunicodechar{Π}{\ensuremath{\Pi}}
\newunicodechar{Ω}{\ensuremath{\Omega}}

\hypersetup{
    colorlinks=true,
    linkcolor=janusblue,
    citecolor=janusblue,
    urlcolor=accentgold,
    pdftitle={JANUS Neuromorphic Architecture},
    pdfauthor={Jordan Smith}
}

% --- HEADER & FOOTER ---
\pagestyle{fancy}
\fancyhf{}
\fancyhead[L]{\textbf{JANUS Neuromorphic}}
\fancyhead[R]{\textit{Brain-Inspired Trading}}
\fancyfoot[C]{\thepage}
\renewcommand{\headrulewidth}{0.4pt}
\renewcommand{\footrulewidth}{0pt}

% --- SECTION STYLING ---
\titleformat{\section}
  {\color{neurocolor}\normalfont\Large\bfseries}
  {\thesection}{1em}{}
\titleformat{\subsection}
  {\color{neurocolor}\normalfont\large\bfseries}
  {\thesubsection}{1em}{}
\titleformat{\subsubsection}
  {\color{neurocolor}\normalfont\normalsize\bfseries}
  {\thesubsubsection}{1em}{}

% --- CODE SNIPPET STYLE ---
\lstset{
    basicstyle=\ttfamily\small,
    breaklines=true,
    frame=single,
    backgroundcolor=\color{codegray},
    keywordstyle=\color{blue},
    commentstyle=\color{green!50!black},
    stringstyle=\color{red},
    numbers=left,
    numberstyle=\tiny\color{gray}
}

% --- MATH OPERATORS ---
\DeclareMathOperator*{\argmax}{arg\,max}
\DeclareMathOperator*{\argmin}{arg\,min}
\DeclareMathOperator{\softmax}{softmax}
\DeclareMathOperator{\sigmoid}{sigmoid}

% --- DOCUMENT START ---
\begin{document}

% =============================================================================
% TITLE PAGE
% =============================================================================
\begin{titlepage}
    \pagenumbering{gobble}
    \centering
    \vspace*{3cm}

    {\Huge \textbf{\textcolor{neurocolor}{JANUS}}} \\[0.3cm]
    {\LARGE \textbf{Neuromorphic Architecture}} \\[1.5cm]

    {\Large \textit{Brain-Inspired Algorithmic Trading System}} \\[0.3cm]
    {\Large \textit{Mapping Neuroscience to Market Intelligence}} \\[3cm]

    \textbf{\Large Classification: Architecture Specification} \\[0.5cm]
    \textbf{\Large Version: 1.0} \\[3cm]

    \textbf{Author:} Jordan Smith \\
    \textit{github.com/nuniesmith} \\[0.5cm]
    \textbf{Date:} \today

    \vfill
    \begin{tcolorbox}[colback=codegray, colframe=braincolor, width=0.85\textwidth]
    \centering
    \textbf{Neuromorphic Design Principles:}
    \begin{itemize}[leftmargin=*]
        \item \textbf{Cognitive Mapping:} Each brain region maps to a trading subsystem
        \item \textbf{Hierarchical Processing:} From sensory input to strategic planning
        \item \textbf{Parallel Computation:} Multiple regions process simultaneously
        \item \textbf{Homeostatic Regulation:} Self-balancing risk and reward
        \item \textbf{Fear-Conditioned Safety:} Emotional override for threat response
    \end{itemize}
    \end{tcolorbox}
    \vfill
\end{titlepage}

% =============================================================================
% ABSTRACT
% =============================================================================
\newpage
\pagenumbering{arabic}
\thispagestyle{plain}
\section*{Abstract}

JANUS implements a \textbf{neuromorphic architecture} that maps cognitive neuroscience principles to algorithmic trading. Each brain region's computational role is replicated in the system architecture, creating a biologically-inspired trading intelligence that combines:

\begin{itemize}
    \item \textbf{Cortex}: Strategic planning and long-term memory (Manager Agent)
    \item \textbf{Hippocampus}: Episodic memory and experience replay (Worker Agent)
    \item \textbf{Basal Ganglia}: Action selection via Actor-Critic RL
    \item \textbf{Thalamus}: Attention gating and multimodal fusion
    \item \textbf{Prefrontal Cortex}: Logic, planning, and regulatory compliance
    \item \textbf{Amygdala}: Fear detection and emergency circuit breakers
    \item \textbf{Hypothalamus}: Homeostatic risk regulation
    \item \textbf{Cerebellum}: Motor control and optimal execution
    \item \textbf{Visual Cortex}: Pattern recognition via GAF and ViViT
    \item \textbf{Integration}: Brainstem coordination and lifecycle management
\end{itemize}

This architecture enables emergent intelligence through parallel processing, hierarchical control, and homeostatic self-regulation—principles proven effective in biological systems over millions of years of evolution.

\newpage
\tableofcontents
\newpage

% =============================================================================
% SECTION 1: NEUROMORPHIC PHILOSOPHY
% =============================================================================
\section{Neuromorphic Design Philosophy}
\label{sec:philosophy}

\subsection{Why Brain-Inspired Architecture?}

Traditional trading systems follow rigid, hierarchical designs. JANUS instead adopts principles from cognitive neuroscience:

\begin{enumerate}
    \item \textbf{Parallel Processing}: Multiple brain regions process different aspects of market data simultaneously
    \item \textbf{Hierarchical Abstraction}: Low-level pattern recognition feeds mid-level tactics which inform high-level strategy
    \item \textbf{Homeostatic Regulation}: Self-balancing mechanisms maintain system health (like biological homeostasis)
    \item \textbf{Emotional Override}: Fear systems can immediately halt trading when threats are detected
    \item \textbf{Memory Consolidation}: Wake-sleep cycles transfer episodic experiences to long-term schemas
    \item \textbf{Adaptive Learning}: Continuous learning at multiple timescales (fast hippocampal, slow cortical)
\end{enumerate}

\subsection{Neuroscience-to-Trading Mapping}

\begin{table}[H]
\centering
\begin{tabularx}{\textwidth}{|l|L|L|}
\hline
\textbf{Brain Region} & \textbf{Neuroscience Function} & \textbf{Trading Function} \\
\hline
\textbf{Cortex} & Executive function, strategic planning, declarative memory & Manager agent, portfolio strategy, market knowledge \\
\hline
\textbf{Hippocampus} & Episodic memory, spatial navigation, memory replay & Worker agent, trade history, experience replay \\
\hline
\textbf{Basal Ganglia} & Action selection, habit formation, reward learning & Actor-Critic RL, action gating, Q-learning \\
\hline
\textbf{Thalamus} & Sensory relay, attention gating, arousal & Data fusion, attention mechanisms, signal filtering \\
\hline
\textbf{Prefrontal} & Logic, planning, impulse control, ethics & LTN constraints, compliance, goal decomposition \\
\hline
\textbf{Amygdala} & Fear conditioning, threat detection, emotional memory & Risk detection, circuit breakers, kill switch \\
\hline
\textbf{Hypothalamus} & Homeostasis, motivation, energy balance & Risk appetite, position sizing, cash management \\
\hline
\textbf{Cerebellum} & Motor coordination, procedural learning, error correction & Order execution, slippage prediction, PID control \\
\hline
\textbf{Visual Cortex} & Visual processing, feature extraction, object recognition & GAF encoding, ViViT, pattern recognition \\
\hline
\textbf{Brainstem} & Basic life functions, arousal/sleep cycles & System orchestration, wake/sleep coordination \\
\hline
\end{tabularx}
\caption{Neuroscience to Trading Mapping}
\end{table}

% =============================================================================
% SECTION 2: BRAIN REGION ARCHITECTURES
% =============================================================================
\section{Brain Region Architectures}
\label{sec:regions}

\subsection{Cortex: Strategic Planning \& Long-term Memory}

\subsubsection{Neuroscience Background}
The cerebral cortex handles executive function, strategic planning, and declarative (fact-based) memory. It operates on slow timescales, consolidating knowledge over days to years.

\subsubsection{Trading Implementation}

\textbf{Directory:} \texttt{src/janus/neuromorphic/cortex/}

\textbf{Components:}
\begin{itemize}
    \item \textbf{Manager}: Feudal RL manager agent for high-level strategy
    \item \textbf{Memory}: Long-term knowledge consolidation and schema storage
    \item \textbf{Planning}: Scenario analysis, Monte Carlo, portfolio optimization
\end{itemize}

\textbf{Key Responsibilities:}
\begin{enumerate}
    \item Set strategic goals (e.g., "maximize Sharpe ratio while maintaining drawdown <15\%")
    \item Generate subgoals for Worker agent (e.g., "accumulate position in AAPL over 2 hours")
    \item Consolidate episodic memories into abstract schemas (e.g., "morning volatility regime")
    \item Store declarative knowledge (e.g., "FOMC announcements increase volatility")
\end{enumerate}

\textbf{Mathematical Formulation:}

Manager policy selects subgoals $g$ for Worker:
\begin{equation}
    g_t = \pi_{\text{Manager}}(s_t^{\text{high}})
\end{equation}
where $s_t^{\text{high}}$ is high-level state (portfolio metrics, regime, time-to-horizon).

Value function:
\begin{equation}
    V_{\text{Manager}}(s) = \mathbb{E}\left[\sum_{t=0}^{T} \gamma^t r_t^{\text{high}} \mid s_0 = s\right]
\end{equation}

\subsection{Hippocampus: Episodic Memory \& Experience Replay}

\subsubsection{Neuroscience Background}
The hippocampus rapidly encodes episodic memories and replays them during sleep at 10-20× speed. Pattern separation prevents interference. Sharp Wave Ripples (SWR) prioritize important experiences.

\subsubsection{Trading Implementation}

\textbf{Directory:} \texttt{src/janus/neuromorphic/hippocampus/}

\textbf{Components:}
\begin{itemize}
    \item \textbf{Worker}: Feudal RL worker agent for tactical execution
    \item \textbf{Replay}: Prioritized Experience Replay (PER) buffer
    \item \textbf{Episodes}: Trade sequences and market events
    \item \textbf{SWR}: Sharp Wave Ripple simulator for compressed replay
\end{itemize}

\textbf{Key Responsibilities:}
\begin{enumerate}
    \item Execute subgoals from Manager (e.g., "buy 100 shares incrementally")
    \item Store trade experiences in episodic buffer
    \item Prioritize replay based on TD-error + logic violations
    \item Compress replay 10-20× during sleep (Backward service)
\end{enumerate}

\textbf{Mathematical Formulation:}

Worker policy conditioned on subgoal $g$:
\begin{equation}
    a_t = \pi_{\text{Worker}}(s_t^{\text{low}}, g_t)
\end{equation}

Intrinsic reward for subgoal completion:
\begin{equation}
    r_t^{\text{intrinsic}} = -||s_t - g_t||^2
\end{equation}

Prioritized replay sampling:
\begin{equation}
    P(i) = \frac{p_i^\alpha}{\sum_j p_j^\alpha}, \quad p_i = |\delta_i| + \lambda_{\text{logic}} v_i + \lambda_{\text{reward}} |r_i|
\end{equation}

\subsection{Basal Ganglia: Action Selection \& Reinforcement Learning}

\subsubsection{Neuroscience Background}
The basal ganglia implement action selection via dual pathways: direct (Go) promotes actions, indirect (No-Go) inhibits them. This is the biological substrate for reinforcement learning.

\subsubsection{Trading Implementation}

\textbf{Directory:} \texttt{src/janus/neuromorphic/basal\_ganglia/}

\textbf{Components:}
\begin{itemize}
    \item \textbf{Actor}: Policy network for action distribution
    \item \textbf{Critic}: Value network for advantage estimation
    \item \textbf{Praxeological}: Go/No-Go signal computation
    \item \textbf{Selection}: Competitive action selection mechanisms
\end{itemize}

\textbf{Key Responsibilities:}
\begin{enumerate}
    \item Generate action proposals (BUY, SELL, HOLD, sizes)
    \item Compute action values (Q-values)
    \item Gate actions through dual pathways (safety)
    \item Maintain habit cache for frequent patterns
\end{enumerate}

\textbf{Mathematical Formulation:}

Actor policy:
\begin{equation}
    \pi_\theta(a|s) = \softmax(\mathbf{W}_\pi \mathbf{h}(s) + \mathbf{b}_\pi)
\end{equation}

Critic value estimate:
\begin{equation}
    V_\omega(s) = \mathbf{W}_V \mathbf{h}(s) + b_V
\end{equation}

Advantage:
\begin{equation}
    A(s,a) = Q(s,a) - V(s)
\end{equation}

Go signal (direct pathway):
\begin{equation}
    \text{Go}(a) = \max\left(\mathbf{W}_{\text{direct}} \mathbf{h}(s)\right)_a
\end{equation}

No-Go signal (indirect pathway):
\begin{equation}
    \text{NoGo}(a) = \sigmoid(\mathbf{W}_{\text{indirect}} [\mathbf{h}(s); \text{risk}; \text{VPIN}])
\end{equation}

Final action gate:
\begin{equation}
    a_{\text{final}} = \begin{cases}
        a_{\text{proposed}} & \text{if NoGo}(a) < \tau_{\text{veto}} \\
        \text{HOLD} & \text{otherwise}
    \end{cases}
\end{equation}

\subsection{Thalamus: Attention \& Multimodal Fusion}

\subsubsection{Neuroscience Background}
The thalamus acts as a sensory relay station, gating information flow to cortex based on attention and relevance. It integrates multimodal sensory inputs.

\subsubsection{Trading Implementation}

\textbf{Directory:} \texttt{src/janus/neuromorphic/thalamus/}

\textbf{Components:}
\begin{itemize}
    \item \textbf{Attention}: Cross-attention mechanisms
    \item \textbf{Gating}: Sensory gating and relevance filtering
    \item \textbf{Routing}: Dynamic information routing
    \item \textbf{Fusion}: Price, volume, orderbook, sentiment fusion
\end{itemize}

\textbf{Key Responsibilities:}
\begin{enumerate}
    \item Gate incoming market data (filter noise)
    \item Fuse multiple data modalities (price, volume, text)
    \item Route relevant information to appropriate regions
    \item Implement attention mechanisms for saliency
\end{enumerate}

\textbf{Mathematical Formulation:}

Gated cross-attention:
\begin{equation}
    \text{Attention}(\mathbf{Q}, \mathbf{K}, \mathbf{V}) = \softmax\left(\frac{\mathbf{Q}\mathbf{K}^\top}{\sqrt{d_k}}\right) \mathbf{V}
\end{equation}

Gating scalar:
\begin{equation}
    \lambda_{\text{gate}} = \sigmoid(\mathbf{W}_g [\mathbf{h}_m; \mathbf{h}_n] + b_g)
\end{equation}

Fused representation:
\begin{equation}
    \mathbf{h}_{\text{fused}} = \mathbf{h}_m + \lambda_{\text{gate}} \cdot \text{Attention}(\mathbf{h}_m, \mathbf{h}_n, \mathbf{h}_n)
\end{equation}

\subsection{Prefrontal Cortex: Logic, Planning \& Compliance}

\subsubsection{Neuroscience Background}
The prefrontal cortex implements logical reasoning, impulse control, and ethical decision-making. It's the "executive" that enforces rules and long-term goals.

\subsubsection{Trading Implementation}

\textbf{Directory:} \texttt{src/janus/neuromorphic/prefrontal/}

\textbf{Components:}
\begin{itemize}
    \item \textbf{LTN}: Logic Tensor Networks for rule encoding
    \item \textbf{Conscience}: Compliance constraints (wash sale, risk limits)
    \item \textbf{Planning}: Goal decomposition and plan synthesis
    \item \textbf{Goals}: Goal hierarchy management
\end{itemize}

\textbf{Key Responsibilities:}
\begin{enumerate}
    \item Encode trading rules as differentiable logic
    \item Enforce regulatory compliance (wash sale, position limits)
    \item Block actions violating constraints
    \item Decompose high-level goals into actionable plans
\end{enumerate}

\textbf{Mathematical Formulation:}

LTN predicate grounding:
\begin{equation}
    \mathcal{G}(P)(\mathbf{x}) = \sigmoid(\mathbf{W}_P \mathbf{x} + b_P) \in [0,1]
\end{equation}

Łukasiewicz conjunction:
\begin{equation}
    \mathcal{G}(A \land B) = \max(0, \mathcal{G}(A) + \mathcal{G}(B) - 1)
\end{equation}

Wash sale constraint:
\begin{equation}
    \forall t, k \in [1,30]: \neg(\text{SaleAtLoss}(t) \land \text{Buy}(t+k))
\end{equation}

Satisfiability:
\begin{equation}
    \text{SatAgg}(\mathcal{K}) = \left(\frac{1}{m}\sum_{i=1}^m \mathcal{G}(\phi_i)^p\right)^{1/p}
\end{equation}

\subsection{Amygdala: Fear, Threat Detection \& Circuit Breakers}

\subsubsection{Neuroscience Background}
The amygdala detects threats and triggers immediate fear responses, overriding rational planning when danger is present. Fear-conditioned memories persist long-term.

\subsubsection{Trading Implementation}

\textbf{Directory:} \texttt{src/janus/neuromorphic/amygdala/}

\textbf{Components:}
\begin{itemize}
    \item \textbf{Fear}: Fear-conditioned inhibition network (FNI-RL)
    \item \textbf{VPIN}: Volume-synchronized toxicity detection
    \item \textbf{Circuit Breakers}: Kill switch, position freeze, cancel all
    \item \textbf{Threat Detection}: Anomaly, flash crash, black swan detection
\end{itemize}

\textbf{Key Responsibilities:}
\begin{enumerate}
    \item Detect market panic and flash crashes
    \item Trigger emergency circuit breakers
    \item Override all other systems in extreme conditions
    \item Learn fear-conditioned responses to past disasters
\end{enumerate}

\textbf{Mathematical Formulation:}

VPIN (Volume-Synchronized Probability of Informed Trading):
\begin{equation}
    \text{VPIN}_t = \frac{\sum_{i=1}^n |V_{\text{buy},i} - V_{\text{sell},i}|}{\sum_{i=1}^n V_i}
\end{equation}

Fear activation:
\begin{equation}
    f_{\text{fear}}(s) = \sigmoid(\mathbf{W}_f [\text{VPIN}; \sigma_{\text{vol}}; \Delta p_{\text{max}}] + b_f)
\end{equation}

Circuit breaker trigger:
\begin{equation}
    \text{KillSwitch} = \begin{cases}
        \text{ACTIVATE} & \text{if } f_{\text{fear}} > \tau_{\text{fear}} \text{ OR VPIN} > \tau_{\text{VPIN}} \\
        \text{STANDBY} & \text{otherwise}
    \end{cases}
\end{equation}

\subsection{Hypothalamus: Homeostasis \& Risk Appetite}

\subsubsection{Neuroscience Background}
The hypothalamus maintains homeostasis—internal balance of temperature, hunger, thirst, etc. It regulates motivation and energy expenditure.

\subsubsection{Trading Implementation}

\textbf{Directory:} \texttt{src/janus/neuromorphic/hypothalamus/}

\textbf{Components:}
\begin{itemize}
    \item \textbf{Homeostasis}: Balance tracking and deviation correction
    \item \textbf{Position Sizing}: Kelly criterion, volatility scaling
    \item \textbf{Risk Appetite}: Dynamic risk tolerance adaptation
    \item \textbf{Energy}: Capital allocation and cash reserves
\end{itemize}

\textbf{Key Responsibilities:}
\begin{enumerate}
    \item Maintain target portfolio balance (setpoints)
    \item Adjust position sizes based on volatility and drawdown
    \item Regulate risk appetite (fear vs. greed)
    \item Ensure cash reserves and leverage limits
\end{enumerate}

\textbf{Mathematical Formulation:}

Kelly criterion (fractional):
\begin{equation}
    f^* = \frac{p(b+1) - 1}{b}, \quad \text{position size} = \frac{f^*}{2} \cdot \text{capital}
\end{equation}

Volatility scaling:
\begin{equation}
    \text{size}_{\text{adjusted}} = \text{size}_{\text{base}} \cdot \frac{\sigma_{\text{target}}}{\sigma_{\text{current}}}
\end{equation}

Drawdown scaling:
\begin{equation}
    \text{size}_{\text{DD}} = \text{size}_{\text{base}} \cdot \max\left(0.1, 1 - \frac{\text{DD}_{\text{current}}}{\text{DD}_{\text{max}}}\right)
\end{equation}

Homeostatic correction:
\begin{equation}
    \Delta \text{allocation} = K_p \cdot (\text{target} - \text{current}) + K_d \cdot \frac{d(\text{target} - \text{current})}{dt}
\end{equation}

\subsection{Cerebellum: Motor Control \& Execution}

\subsubsection{Neuroscience Background}
The cerebellum coordinates fine motor control, learns procedural skills, and predicts sensory consequences of actions (forward models).

\subsubsection{Trading Implementation}

\textbf{Directory:} \texttt{src/janus/neuromorphic/cerebellum/}

\textbf{Components:}
\begin{itemize}
    \item \textbf{Execution}: Order routing, TWAP/VWAP algorithms
    \item \textbf{Impact}: Almgren-Chriss optimal execution
    \item \textbf{Forward Models}: Latency compensation, fill prediction
    \item \textbf{Error Correction}: PID control, adaptive feedback
\end{itemize}

\textbf{Key Responsibilities:}
\begin{enumerate}
    \item Route orders to exchanges with minimal slippage
    \item Predict and minimize market impact
    \item Compensate for execution latency (Smith predictor)
    \item Learn from execution errors and adapt
\end{enumerate}

\textbf{Mathematical Formulation:}

Almgren-Chriss optimal trajectory:
\begin{equation}
    x_t = X \cdot \frac{\sinh(\kappa(T-t))}{\sinh(\kappa T)}, \quad \kappa = \sqrt{\frac{\eta \sigma}{\tau}}
\end{equation}

Market impact:
\begin{equation}
    \text{Impact} = \eta \cdot \sigma \cdot \sqrt{\frac{v}{V_{\text{avg}}}}
\end{equation}

Smith predictor (latency compensation):
\begin{equation}
    u(t) = K_c \left[e(t) + \frac{1}{\tau_I}\int e(\tau)d\tau + \tau_D \frac{de(t)}{dt}\right] + \hat{p}(t + \Delta t)
\end{equation}

Execution error:
\begin{equation}
    \epsilon_{\text{exec}} = |p_{\text{actual}} - p_{\text{predicted}}|
\end{equation}

\subsection{Visual Cortex: Pattern Recognition \& Vision}

\subsubsection{Neuroscience Background}
The visual cortex processes images hierarchically: V1 detects edges, V2 detects shapes, V4 detects objects. It implements hierarchical feature extraction.

\subsubsection{Trading Implementation}

\textbf{Directory:} \texttt{src/janus/neuromorphic/visual\_cortex/}

\textbf{Components:}
\begin{itemize}
    \item \textbf{Eyes}: Data ingestion, preprocessing, streaming
    \item \textbf{GAF}: Gramian Angular Fields (GASF, GADF, DiffGAF)
    \item \textbf{ViViT}: Video Vision Transformer for spatiotemporal patterns
    \item \textbf{Visualization}: UMAP, GradCAM, saliency maps
\end{itemize}

\textbf{Key Responsibilities:}
\begin{enumerate}
    \item Ingest and preprocess raw market data
    \item Transform time series to visual manifolds (GAF)
    \item Extract spatiotemporal patterns (ViViT)
    \item Visualize learned representations (UMAP)
\end{enumerate}

\textbf{Mathematical Formulation:}

GAF normalization:
\begin{equation}
    \tilde{x}_i = \tanh\left(\frac{x_i - \min(X)}{\max(X) - \min(X) + \epsilon} \cdot \alpha + \beta\right)
\end{equation}

GASF:
\begin{equation}
    \text{GASF}_{i,j} = \cos(\phi_i + \phi_j), \quad \phi_i = \arccos(\tilde{x}_i)
\end{equation}

GADF:
\begin{equation}
    \text{GADF}_{i,j} = \sin(\phi_i - \phi_j)
\end{equation}

ViViT patch embedding:
\begin{equation}
    \mathbf{z}_{f,i,j}^{(0)} = \mathbf{E} \cdot \text{flatten}(\mathcal{V}_{f,i:i+P,j:j+P}) + \mathbf{p}_{f,i,j}
\end{equation}

\subsection{Integration: Brainstem \& Global Coordination}

\subsubsection{Neuroscience Background}
The brainstem controls basic life functions, arousal/sleep cycles, and global state coordination. It's the "operating system" of the brain.

\subsubsection{Trading Implementation}

\textbf{Directory:} \texttt{src/janus/neuromorphic/integration/}

\textbf{Components:}
\begin{itemize}
    \item \textbf{Workflow}: State machine orchestration
    \item \textbf{State}: Global state management, message bus
    \item \textbf{API}: REST, gRPC, WebSocket interfaces
    \item \textbf{Engine}: Wake-sleep cycle coordination
\end{itemize}

\textbf{Key Responsibilities:}
\begin{enumerate}
    \item Coordinate wake (Forward) and sleep (Backward) cycles
    \item Manage global system state
    \item Route messages between brain regions
    \item Expose external APIs
\end{enumerate}

% =============================================================================
% SECTION 3: INFORMATION FLOW
% =============================================================================
\section{Information Flow Diagrams}
\label{sec:flow}

\subsection{Wake State (Forward Service)}

\begin{verbatim}
Market Data → Visual Cortex (GAF/ViViT) → Thalamus (Fusion)
                                               ↓
                                          Cortex (Manager)
                                               ↓
                                          Hippocampus (Worker)
                                               ↓
                                          Basal Ganglia (Actor-Critic)
                                               ↓
                                          Prefrontal (LTN Check)
                                               ↓
                                          Amygdala (Fear Check) ──→ Circuit Breaker?
                                               ↓
                                          Hypothalamus (Size Adjust)
                                               ↓
                                          Cerebellum (Execute Order)
                                               ↓
                                          Exchange
\end{verbatim}

\subsection{Sleep State (Backward Service)}

\begin{verbatim}
Hippocampus (Episodic Buffer) → SWR Replay (10-20x speed)
                                      ↓
                                 Prioritized Sampling
                                      ↓
                                 Basal Ganglia (Update Critic)
                                      ↓
                                 Prefrontal (Update LTN)
                                      ↓
                                 Cortex (Schema Consolidation)
                                      ↓
                                 Long-term Memory (Qdrant)
\end{verbatim}

% =============================================================================
% SECTION 4: IMPLEMENTATION GUIDE
% =============================================================================
\section{Implementation Guide}
\label{sec:implementation}

\subsection{Directory Structure}

\begin{verbatim}
src/janus/neuromorphic/
├── lib.rs                    # Main library entry point
├── common/                   # Shared types and utilities
├── cortex/                   # Strategic planning & LTM
│   ├── manager/             # Feudal RL manager
│   ├── memory/              # Consolidation, schemas
│   └── planning/            # Scenario analysis
├── hippocampus/             # Episodic memory & replay
│   ├── worker/              # Feudal RL worker
│   ├── replay/              # PER buffer
│   ├── episodes/            # Trade sequences
│   └── swr/                 # Sharp wave ripples
├── basal_ganglia/           # Action selection & RL
│   ├── actor/               # Policy network
│   ├── critic/              # Value network
│   ├── praxeological/       # Go/No-Go signals
│   └── selection/           # Action selection
├── thalamus/                # Attention & fusion
│   ├── attention/           # Cross-attention
│   ├── gating/              # Sensory gates
│   ├── routing/             # Information routing
│   └── fusion/              # Multimodal fusion
├── prefrontal/              # Logic & compliance
│   ├── ltn/                 # Logic Tensor Networks
│   ├── conscience/          # Compliance rules
│   ├── planning/            # Goal decomposition
│   └── goals/               # Goal management
├── amygdala/                # Fear & circuit breakers
│   ├── fear/                # FNI-RL network
│   ├── vpin/                # Toxicity detection
│   ├── circuit_breakers/    # Kill switch
│   └── threat_detection/    # Anomaly detection
├── hypothalamus/            # Homeostasis & risk
│   ├── homeostasis/         # Balance tracking
│   ├── position_sizing/     # Kelly, vol scaling
│   ├── risk_appetite/       # Dynamic tolerance
│   └── energy/              # Capital allocation
├── cerebellum/              # Motor control & execution
│   ├── execution/           # Order routing
│   ├── impact/              # Almgren-Chriss
│   ├── forward_models/      # Latency compensation
│   └── error_correction/    # PID control
├── visual_cortex/           # Pattern recognition
│   ├── eyes/                # Data ingestion
│   ├── gaf/                 # GAF transformation
│   ├── vivit/               # ViViT model
│   └── visualization/       # UMAP, GradCAM
└── integration/             # System coordination
    ├── workflow/            # State machines
    ├── state/               # Global state
    ├── api/                 # External APIs
    └── engine/              # Orchestration
\end{verbatim}

\subsection{Implementation Checklist}

\begin{enumerate}
    \item \textbf{Phase 1: Core Infrastructure (Weeks 1-2)}
    \begin{itemize}
        \item[$\square$] Set up neuromorphic module structure
        \item[$\square$] Implement common types and error handling
        \item[$\square$] Create inter-region message bus
        \item[$\square$] Set up integration/engine orchestrator
    \end{itemize}

    \item \textbf{Phase 2: Visual Processing (Weeks 3-4)}
    \begin{itemize}
        \item[$\square$] Implement Visual Cortex data ingestion
        \item[$\square$] Implement GAF transformation (GASF, GADF)
        \item[$\square$] Integrate ViViT model (ONNX or tch-rs)
        \item[$\square$] Add UMAP visualization
    \end{itemize}

    \item \textbf{Phase 3: Decision Making (Weeks 5-7)}
    \begin{itemize}
        \item[$\square$] Implement Basal Ganglia Actor-Critic
        \item[$\square$] Implement Prefrontal LTN constraints
        \item[$\square$] Implement Thalamus fusion mechanisms
        \item[$\square$] Connect visual → decision pipeline
    \end{itemize}

    \item \textbf{Phase 4: Memory Systems (Weeks 8-10)}
    \begin{itemize}
        \item[$\square$] Implement Hippocampus episodic buffer
        \item[$\square$] Implement Prioritized Experience Replay
        \item[$\square$] Implement Cortex schema consolidation
        \item[$\square$] Implement SWR compressed replay
    \end{itemize}

    \item \textbf{Phase 5: Safety \& Control (Weeks 11-12)}
    \begin{itemize}
        \item[$\square$] Implement Amygdala fear network
        \item[$\square$] Implement circuit breakers and kill switch
        \item[$\square$] Implement Hypothalamus homeostasis
        \item[$\square$] Implement Cerebellum execution control
    \end{itemize}

    \item \textbf{Phase 6: Integration \& Testing (Weeks 13-14)}
    \begin{itemize}
        \item[$\square$] Connect all brain regions
        \item[$\square$] Implement wake-sleep cycle coordination
        \item[$\square$] End-to-end integration tests
        \item[$\square$] Performance optimization
    \end{itemize}
\end{enumerate}

% =============================================================================
% SECTION 5: ARCHITECTURAL INVARIANTS
% =============================================================================
\section{Architectural Invariants}
\label{sec:invariants}

\subsection{Safety-Critical Invariants}

\begin{enumerate}
    \item \textbf{Amygdala Override}: Fear system ALWAYS overrides all other regions
    \item \textbf{Prefrontal Veto}: LTN constraints MUST block non-compliant actions
    \item \textbf{No Panic}: No \texttt{panic!()}, \texttt{unwrap()}, or \texttt{expect()} in production code
    \item \textbf{Fail-Safe}: Circuit breakers must be fail-safe (default to HALT)
    \item \textbf{Homeostasis}: Cash reserves must never fall below 20\%
\end{enumerate}

\subsection{Performance Invariants}

\begin{enumerate}
    \item \textbf{Forward Latency}: Visual Cortex → Decision <100ms
    \item \textbf{Backward Throughput}: Process >10k experiences per sleep cycle
    \item \textbf{Memory Efficiency}: Hippocampal buffer <10k transitions (FIFO eviction)
    \item \textbf{Parallel Processing}: Brain regions process concurrently
\end{enumerate}

\subsection{Learning Invariants}

\begin{enumerate}
    \item \textbf{Dual Timescale}: Fast hippocampal learning, slow cortical consolidation
    \item \textbf{Recall Gating}: Cortical updates gated by recall strength AND logic validity
    \item \textbf{Priority Replay}: Replay prioritized by TD-error + logic violations + reward
    \item \textbf{Schema Formation}: Clusters detected via UMAP + DBSCAN
\end{enumerate}

% =============================================================================
% BIBLIOGRAPHY
% =============================================================================
\newpage
\begin{thebibliography}{99}
\raggedright

\bibitem{janus_forward} Jordan Smith, "JANUS Forward: Wake State Logic Trading Algorithm," 2025.

\bibitem{janus_backward} Jordan Smith, "JANUS Backward: Sleep State Memory Management," 2025.

\bibitem{feudal_rl} Dayan, Hinton, "Feudal Reinforcement Learning," NIPS 1992.

\bibitem{actor_critic} Sutton, Barto, "Reinforcement Learning: An Introduction," 2nd Ed., 2018.

\bibitem{ltn} Badreddine et al., "Logic Tensor Networks," Artificial Intelligence, 2022.

\bibitem{per} Schaul et al., "Prioritized Experience Replay," ICLR 2016.

\bibitem{swr} Foster, Wilson, "Reverse Replay of Behavioural Sequences in Hippocampal Place Cells," Nature 2006.

\bibitem{fear_rl} "Fear-Conditioned Inhibition in Reinforcement Learning," 2020.

\bibitem{homeostasis} Sterling, Eyer, "Allostasis: A New Paradigm to Explain Arousal Pathology," 1988.

\bibitem{almgren_chriss} Almgren, Chriss, "Optimal Execution of Portfolio Transactions," Journal of Risk, 2000.

\bibitem{vpin} Easley et al., "Flow Toxicity and Liquidity in a High-frequency World," Review of Financial Studies, 2012.

\bibitem{gaf} Wang, Oates, "Encoding Time Series as Images," AAAI 2015.

\bibitem{vivit} Arnab et al., "ViViT: A Video Vision Transformer," ICCV 2021.

\bibitem{umap} McInnes et al., "UMAP: Uniform Manifold Approximation and Projection," 2018.

\bibitem{neuroscience} Kandel et al., "Principles of Neural Science," 6th Ed., 2021.

\end{thebibliography}

\end{document}
